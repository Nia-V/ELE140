%=================================================================
% Modèle de devoir - Version française
% Auteur : <Votre nom>
% Description : Modèle LaTeX pour soumissions de devoirs
%=================================================================

\documentclass[12.0 pt,a4paper]{article}


%---------------------------------------------------------------
% PACKAGES
%---------------------------------------------------------------
\usepackage[T1]{fontenc}              % Encodage correct
\usepackage[utf8]{inputenc}           % Support UTF-8
\usepackage[french]{babel}            % Langue française
\usepackage{geometry}                 % Marges
\geometry{margin=2.0cm}
\usepackage{ amssymb }
\usepackage{amsmath,amssymb,amsthm}   % Math
\usepackage{graphicx}                 % Images
\usepackage{array}                    % Tableaux avancés
\usepackage{tikz}                     % K-maps, circuits logiques
\usetikzlibrary{matrix, positioning}  % Bibliothèques TikZ nécessaires
\usepackage{caption}                  % Légendes personnalisées
\usepackage{multirow}                 % Cellules multi-lignes
\usepackage{xcolor}                   % Couleurs
\usepackage{fancyhdr}                 % En-têtes / pieds de page
\usepackage{hyperref}                 % Liens hypertextes
\usepackage{float}                    % Placement précis
\usepackage{datetime}                 % Formatage de dates
\usepackage{ctable}
\usetikzlibrary{calc}
\usepackage{tabu}
\usetikzlibrary{shapes, arrows, positioning}
\usepackage{changepage} % allows local margin changes
\usepackage{subfigure}
%---------------------------------------------------------------
% INFORMATIONS DU COURS ET DU GROUPE
%---------------------------------------------------------------
\newcommand{\codeclasse}{ELE140}                     % Code du cours
\newcommand{\nomclasse}{Conception des systèmes numériques} % Nom du cours
\newcommand{\titredevoir}{Laboratoire 2} % Titre du devoir
\newcommand{\groupe}{-01}
\newcommand{\enseignant}{Myiah Catwell}
\newcommand{\datedepot}{\today}
\newcommand{\Session}{A25}
\newcommand{\auteurs}{
	Ourania Voyatzis~(VOYO78260401)\\
	Jhermain Louis-Jean~(LOUJ67360401)}

%---------------------------------------------------------------
% MISE EN PAGE
%---------------------------------------------------------------
\pagestyle{fancy}
\fancyhead[L]{\codeclasse\groupe}
\fancyhead[C]{\titredevoir}
\fancyhead[R]{\thepage}
\fancyfoot{}
%---------------------------------------------------------------
% DÉBUT DU DOCUMENT
%---------------------------------------------------------------
\begin{document}
	
	\begin{titlepage}
		\vspace*{\fill} 
		\centering
		\begin{center}
		\includegraphics[width=0.20\textwidth]{Logo_ETS_Blanc_SansTypo_FR}
		\end{center}
		\textbf{\nomclasse\ (\codeclasse\groupe)}\\[0.5em]
		\textbf{\titredevoir}\\[0.5em]
		\textbf{Remis à:} \enseignant \\[1em]
		\textbf{Remis par :}\\[0.5em]
		\auteurs\\ [1.0em]
		\textbf{École de technologie supérieure}\\
		\textbf{Date :} \datedepot
		\vspace*{\fill}
	\end{titlepage}
	
	\tableofcontents
	\newpage
	%---------------------------------------------------------------
	% CONTENU
	%---------------------------------------------------------------
	\section{Introduction (Voyatzis)}
	% In this lab, the task is to create an iterative circuit that can perform a bitwise comparison on  two signed 4 bit integers (A[3..0] and B[3..0]) coded in 2's complement. The scope allows us to use and define 2 carry over signals and to (if nessecary) create a final output function that must have only 3 output signals : PP (A<B), PG (A>B), EG(A = B). 
	
	\subparagraph{}Dans le cadre de cette laboratoire, la tâche consiste à créer un circuit itératif capable d'effectuer une comparaison bit à bit entre deux entiers signés de 4 bits (A[3..0] et B[3..0]) codés en complément à deux. Le cahier des charges nous autorise à utiliser et à définir deux signaux de retenue et à créer (si nécessaire) une fonction de sortie finale qui ne doit comporter que trois signaux de sortie : PP (A<B), PG (A>B), EG (A=B).
	
	
	\section{Cellule-type (Voyatzis)}
	\subsection{La définition générale d'une cellule-type} 
	% A "cellule type" is a collection combinational boolean functions, or a single function, that is meant to be iterated and instantiated multiple times,and chained together using boundary signals. These chained logic functions allow for more complex multibit operations, that can change the final output, based on the the inputs previous state. These boundary signals act as outputs on one iteration and as inputs on the next, in a sense, its a very rudimentary and inefficent way to implememnt input "memory". An example of this type of logic implemented would be the carry-look ahead adder, whose primary inputs are 2 binary digits of the same index and whose boundary signals are Pi and Gi. These boundary signals represent whether the cell generates, and/or propogates a carry over based on the primary inputs prevoious index and curent index.
	\subparagraph{}Une cellule-type est un ensemble de fonctions booléennes combinatoires, ou une seule fonction, destinée à être itérée et instanciée plusieurs fois, puis inter-connectée à l'aide de signaux inter cellule (copy-paste). Ces fonctions logiques identiques et chaînées permettent de réaliser des opérations multibits plus complexes, capables de modifier la sortie finale en fonction de l'état/l'index précédent des entrées. Ces signaux inter cellules agissent comme des sorties lors d'une itération et comme des entrées lors de l'itération suivante; il s'agit, en quelque sorte, d'une manière très rudimentaire et inefficace d'implémenter une \og mémoire \fg~d'entrée primaires.\\\\
	Un exemple de mise en \oe uvre de ce type de logique est l'additionneur \og carry-lookahead\fg, dont les entrées principales sont deux chiffres binaires de même indice et dont les signaux de sortie sont Pi et Gi. Ces signaux indiquent si la cellule génère et/ou propage une retenue en fonction des entrées principales des indices précédents et actuels.
	\subsection{La définition spécifique de la cellule-type pour comparaison} 
	% The definition of the generic cell that can iterate over any number of bits and compare 2 numbers coded in 2's complement, so long as they start the sequence at the MSB. This cell definition has certain conditions that must be respected:
	% - We assume that A and B were always equal before  start up of the machine (A = B  = Z)
	% - if A and B are equal, and were always equal, then Xout != Yout. 
	% - if A was 1 and B was 0 on the first bit (sign bit), for this and every subsequent iteration, Xout = Yout = 0 (A<B => 0=Xout=Yout)
	% - if A was 1 and B was 0 on the any other bit that the first, for this and every subsequent iteration, Xout = Yout = 1 (A>B => 1=Xout=Yout)
	% - if for the  2 above conditions, B is switched with A and vice versa, then the output would be inverted
	
	\subparagraph{}La définition de la cellule générique capable d'itérer sur un nombre quelconque de bits et de comparer deux nombres codés en complément à deux, à condition que la séquence commence par le bit de poids fort (MSB). Cette définition de cellule est soumise à certaines conditions :\\\\
	-Nous supposons que A et B étaient égaux juste avant le démarrage de la machine (A=B=Z).\\
	-Si A et B sont égaux et l'ont toujours été, alors Xout!=Yout.\\
	-Si A vaut 1 et B vaut 0 sur le premier bit (bit de signe), pour cette itération et toutes les itérations suivantes, Xout = Yout = 0 (A < B => 0 = Xout = Yout).\\
	-Si A vaut 1 et B vaut 0 sur un autre bit que le premier, pour cette itération et toutes les itérations suivantes, Xout = Yout = 1 (A > B => 1 = Xout = Yout).\\ 
	- Si, pour les deux conditions ci-dessus, B est interverti avec A et vice versa, alors la sortie est inversée.\\
	\begin{figure}[H]
		\centering
		\begin{tikzpicture}[
			block/.style={draw, thick, minimum width=2.4cm, minimum height=1.4cm, align=center, font=\small},
			signal/.style={->, thick}
			]
			
			%----- Single Block -----
			\node[block] (B) {\textbf{Cellule Type}\\
				$\overline{A_{i}}Y_{in_i}+ B_{i}Y_{in_i}+A_{i}X_{in_i}+\overline{B_{i}}X_{in_i} = X_{out_i}$\\
				$\overline{A_i}B_iY_{in_i}+ A_i\overline{B_{i}}X_{in_i}+X_{in_i}Y_{in_i} = Y_{out_i}$\\ };
			
			%----- A and B inputs from top -----
			\draw[signal, <-] ($(B.north) + (-0.5,0)$) -- ++(0,1.0)
			node[above] {$A_i$};
			\draw[signal, <-] ($(B.north) + (0.5,0)$) -- ++(0,1.0)
			node[above] {$B_i$};
			
			%----- Carry inputs (Xin, Yin) from left -----
			\draw[signal, <-] ([yshift=8pt]B.west) -- ++(-1.2,0)
			node[left]{$X_{in_i}$};
			\draw[signal, <-] ([yshift=-8pt]B.west) -- ++(-1.2,0)
			node[left]{$Y_{in_i}$};
			
			%----- Carry outputs (Xin, Yin) to right -----
			\draw[signal] ([yshift=8pt]B.east) -- ++(1.2,0)
			node[right]{$X_{out_i}$};
			\draw[signal] ([yshift=-8pt]B.east) -- ++(1.2,0)
			node[right]{$Y_{out_i}$};
			
		\end{tikzpicture}
			\caption{\label{fig:figure1} La i-ième cellule-type}
	\end{figure}
	\newpage
	\section{Schéma fonctionnel du comparateur (Voyatzis)}
	\subsection{La définition des signaux inter cellules}
	\subparagraph{}Les signaux intercellulaires sont les signaux qui circulent entre les cellules répétées ; ils transportent des informations codées sur les entrées primaires des cellules précédentes. La signification de chaque signal varie en fonction de l'architecture de la cellule de base répétée.\\\\
	\subsection{La définition des signaux inter cellules $X_{out}$ et $Y_{out}$}
	\subparagraph{}$X_{in}$ et $Y_{in}$ doivent être initialisés respectivement à 0 et 1. En effet, nous devons supposer que $A_{N+1}$ et $B_{N+1}$ seront égaux et à haute impédance au démarrage de la machine.(INIT/SIGN BIT)\\\\Les signaux intercellulaires du circuit conçu représentent cumulativement si A est supérieur, inférieur ou égal à B. Comme spécifié dans la table de vérité ci-dessus, lorsque les signaux intercellulaires $X_{out}$ et $Y_{out}$ sont égaux à 11, cela signifie A > B. Inversement, lorsqu'ils sont égaux à 00, A < B. Lorsque  $X_{out}$ et  $X_{out}$ sont différents ( $X_{out}$ != $Y_{out}$), cela signifie que jusqu'à présent, tous les bits précédents de A et B ont été égaux, en commençant par MSB. Si cet état est maintenu jusqu'au dernier bit (LSB), alors les deux nombres A et B sont exactement identiques.
	
	\subsection{Comparateur}
	\begin{figure}[H]
	\begin{adjustwidth}{-0.85cm}{0cm}   % left margin reduced by 2cm
		
		\tikzstyle{block} = [draw, fill=white, rectangle, 
		minimum height=3em, minimum width=6em]
		\tikzstyle{sum} = [draw, fill=white, circle, node distance=1cm]
		\tikzstyle{input} = [coordinate]
		\tikzstyle{output} = [coordinate]
		\tikzstyle{pinstyle} = [pin edge={to-,thin,black}]
		

			\centering
		\begin{tikzpicture}[
			block/.style={draw, thick, minimum width=1.8cm, minimum height=1.0cm, align=left, font=\small},
			signal/.style={->, thick}
			]
			
			%----- Blocks -----
			\node[block, minimum height = 1.5cm] (B3) {Cell 3};
			\node[block, right=1.3cm of B3, minimum height = 1.5cm] (B2) {Cell 2};
			\node[block, right=1.3cm of B2, minimum height = 1.5cm] (B1) {Cell 1};
			\node[block, right=1.3cm of B1, minimum height = 1.5cm] (B0) {Cell 0};
			\node[block, right=1.3cm of B0, minimum height=1.2cm] (B4) {Comparateur\\ \(\overline{X} * \overline{Y}=  PP\)\\\(X * Y = PG\)\\\(X \bigoplus Y = EG\)};
			
			%----- A and B inputs from the top (horizontally offset, same height) -----
			\foreach \i in {3,2,1,0}
			{
				% Ai slightly left of center
				\draw[signal, <-] ($(B\i.north) + (-0.35,0)$) -- ++(0,1.0)
				node[above] {\small{$A_\i$}};
				
				% Bi slightly right of center
				\draw[signal, <-] ($(B\i.north) + (0.35,0)$) -- ++(0,1.0)
				node[above] {\small{$B_\i$}};
			}
			
			%----- Carry-over signals -----
			% Initial Xin / Yin INTO Block 1
			\draw[signal, <-] ([yshift=6pt]B3.west) -- ++(-1,0) node[left]{\small{$X_{in}=0$}};
			\draw[signal, <-] ([yshift=-6pt]B3.west) -- ++(-1,0) node[left]{\small{$Y_{in}=1$}};
			
			
			% Carry between blocks
			\foreach \i/\j in {3/2,2/1,1/0,0/4}
			{
				\draw[signal,->] 
				([yshift=7pt]B\i.east) 
				-- node[above, midway, font=\small]{$X_{out_{\i}}$} 
				([yshift=7pt]B\j.west);
				
				\draw[signal] 
				([yshift=-9pt]B\i.east) 
				-- node[above, midway, font=\small] {$Y_{out_{\i}}$} 
				([yshift=-9pt]B\j.west);
			}
			
			
			%----- Separate output signals PP, PG, EG -----
			\draw[signal] ($(B4.east) + (0, 12pt)$) -- ++(1.2,0) node[right]{\small{PP}};
			\draw[signal] (B4.east)              -- ++(1.2,0) node[right]{\small{PG}};
			\draw[signal] ($(B4.east) + (0,-12pt)$) -- ++(1.2,0) node[right]{\small{EG}};
			
		\end{tikzpicture}
		\caption{\label{fig:figure2} Schéma fonctionnel du comparateur}

					
	\end{adjustwidth}
		\end{figure}
	
	\section{Table de vérité pour la cellule-type (Voyatzis)}
	\subsection{La définition générale d'une table de vérité}
	% A truth table is a method of encoding and visualizing logic functions. Incrementally changing the inputs and showing the resulting output(s)éééé
	\subparagraph{}Une table de vérité est une méthode permettant de coder et de visualiser des fonctions logiques. Elle consiste à modifier progressivement les entrées et à afficher la ou les sorties correspondantes.\\
	\subsection{Table de vérité}
	\begin{figure}[H]
		\centering
			\caption{\label{fig:figure3} Table de vérité pour la cellule-type: $X_{out}$ et $Y_{out}$}
		\Large\begin{tabu}{|c|c|[1.5pt]cccc|[1.5pt]c|c|}
			\hline
			& No. & $X_{in}$ & $Y_{in}$ & $A_i$ & $B_i$  & $X_{out}$ & $Y_{out}$ \\
			\tabucline[1.5pt]{-}\normalsize\parbox[t]{3.5mm}{\multirow{4}{*}{\rotatebox[origin=c]{90}{Set A<B}}}
			&0  & 0 & 0 & 0 & 0  & 0 & 0 \\&
			1  & 0 & 0 & 0 & 1  & 0 & 0 \\&% 1
			2  & 0 & 0 & 1 & 0  & 0 & 0 \\&% 2
			3  & 0 & 0 & 1 & 1  & 0 & 0 \\\tabucline[1.5pt]{-}\normalsize\parbox[t]{3.5mm}{\multirow{4}{*}{\rotatebox[origin=c]{90}{Bit de signe/init}}}&
			 4  & 0 & 1 & 0 & 0  & 1 & 0 \\&% -4
			5  & 0 & 1 & 0 & 1  & 1 & 1 \\&% -3
			6  & 0 & 1 & 1 & 0  & 0 & 0 \\&% & 0 -27
			7  & 0 & 1 & 1 & 1  & 1 & 0 \\\tabucline[1.5pt]{-}\normalsize\parbox[t]{3.5mm}{\multirow{4}{*}{\rotatebox[origin=c]{90}{Bits suivants}}}&% -1
			8  & 1 & 0 & 0 & 0  & 1 & 0 \\&% 1
			9  & 1 & 0 & 0 & 1  & 0 & 0 \\&% 2
			10 &  1 & 0 & 1 & 0  & 1 & 1 \\& % 3
			11 &  1 & 0 & 1 & 1  & 1 & 0 \\\tabucline[1.5pt]{-}\normalsize\parbox[t]{3.5mm}{\multirow{4}{*}{\rotatebox[origin=c]{90}{Set A>B}}} &% 4
			12 &  1 & 1 & 0 & 0  & 1 & 1 \\ &% -3
			13 &  1 & 1 & 0 & 1  & 1 & 1 \\ &% -2
			14 &  1 & 1 & 1 & 0  & 1 & 1 \\ &% -1
			15 &  1 & 1 & 1 & 1  & 1 & 1 \\\hline % 0
		\end{tabu}
	\end{figure}

	\section{Tables de Karnaugh: $X_{out}$ et $Y_{out}$ (Voyatzis)}
	\subsection{Définition}
	%Karnaugh maps or K-maps are another way of visualizing and encoding logic functions. Instead of incrementally changing the inputs in the standard binary way, for every row and every collumn they increment a grouping of the inputs using gray code. The inputs are split into 2 groups, usually and most easily simplified if the input groupings are equally split. These 2 groupings are split along a vertical and horizontal axis, the incrementented using gray code, creating a square frame that has a sort of coordinate system of input values. Then each cell is filled in with its corresponding output value based on the gray-coded input groupings value at the specific coordinates.
	
%	This method, although counter intuitive at first, allows for functions to be easily simplified to their sum of products/product of sums forms  by grouping the all the ones/zeros on the map, respectively. K-maps even allow us to very quickly correct for static hazards in circuitry, before it is ever drawn/conceptualized. 
	\subparagraph{}Les tableaux de Karnaugh, ou cartes de Karnaugh, constituent une autre méthode de visualisation et de codage des fonctions logiques. Au lieu de modifier progressivement les entrées selon la méthode binaire standard, chaque ligne et chaque colonne incrémentent un groupement d'entrées à l'aide du code de Gray. Les entrées sont divisées en deux groupes, généralement et plus facilement simplifiables si ces groupes sont de taille égale. Ces deux groupes sont répartis le long d'un axe vertical et d'un axe horizontal, l'incrémentation étant effectuée à l'aide du code de Gray, créant ainsi un cadre carré doté d'un système de coordonnées représentant les valeurs d'entrée. Chaque cellule est ensuite remplie avec sa valeur de sortie correspondante, basée sur la valeur des groupements d'entrées codés en code de Gray aux coordonnées spécifiques.\\\\
	Cette méthode, bien qu'intuitivement déroutante au premier abord, permet de simplifier facilement les fonctions sous leurs formes somme de produits/produit de sommes en regroupant respectivement tous les 1 ou tous les 0 sur le tableau. Les cartes de Karnaugh permettent même de corriger très rapidement les aléas statiques dans les circuits, avant même leur conception ou leur schématisation.
	\subsection{K-Maps: $X_{out}$ et $Y_{out}$}
	\begin{figure}[H]
		\centering
		
		% ---------------- KMAP LEFT ----------------
		\begin{minipage}{0.45\textwidth}
			\centering
			\subsection*{} % use subsection* to avoid numbering
			\caption{$X_{out}=\overline{A}Y+ BY+AX+\overline{B}X$}
			\begin{tikzpicture}
				\matrix (m) [matrix of nodes,
				nodes={draw, minimum size=0.75cm, anchor=center},
				column sep=-\pgflinewidth, row sep=-\pgflinewidth] {
					0 & 0 & 0 & 0 \\
					1 & 1 & 1 & 0 \\
					1 & 1 & 1 & 1 \\
					1 & 0 & 1 & 1 \\
				};
				\node[above=4pt of m-1-1] {$\overline{A}\overline{B}$};
				\node[above=4pt of m-1-2] {$\overline{A}B$};
				\node[above=4pt of m-1-3] {$AB$};
				\node[above=4pt of m-1-4] {$A\overline{B}$};
				
				\node[left=4pt of m-1-1, rotate=90, anchor=south] {$\overline{X }\overline{Y}$};
				\node[left=4pt of m-2-1, rotate=90, anchor=south] {$\overline{X}Y$};
				\node[left=4pt of m-3-1, rotate=90, anchor=south] {$XY$};
				\node[left=4pt of m-4-1, rotate=90, anchor=south] {$X\overline{B}$};
			\end{tikzpicture}
			
		\end{minipage}
		\hfill\begin{minipage}{0.45\textwidth}
			\centering
			\subsection*{}
			\caption{$Y_{out}=\overline{A}BY+ A\overline{B}X+XY$}
			\begin{tikzpicture}
				\matrix (m) [matrix of nodes,
				nodes={draw, minimum size=0.75cm, anchor=center},
				column sep=-\pgflinewidth, row sep=-\pgflinewidth] {
					0 & 0 & 0 & 0 \\
					0 & 1 & 0 & 0 \\
					1 & 1 & 1 & 1 \\
					0 & 0 & 0 & 1 \\
				};
				\node[above=4pt of m-1-1] {$\overline{A}\overline{B}$};
				\node[above=4pt of m-1-2] {$\overline{A}B$};
				\node[above=4pt of m-1-3] {$AB$};
				\node[above=4pt of m-1-4] {$A\overline{B}$};
				
				\node[left=4pt of m-1-1, rotate=90, anchor=south] {$\overline{X}\overline{Y}$};
				\node[left=4pt of m-2-1, rotate=90, anchor=south] {$\overline{X}Y$};
				\node[left=4pt of m-3-1, rotate=90, anchor=south] {$XY$};
				\node[left=4pt of m-4-1, rotate=90, anchor=south] {$X\overline{Y}$};
			\end{tikzpicture}
		
		\end{minipage}
		
	\end{figure}
	
	%  I remind you that the neatness of the report will be important: The necessary diagrams and
	% tables should preferably be created on a computer (no photos of
	% hand-drawn diagrams/tables). The use of a tablet will be permitted if the rendering is very clean.
	%  The distribution of tasks within the team is also important (indicate your names in
	% parentheses next to the sections you worked on).
	%  A cover page (course/group, title, school, your names/student IDs, my name,
	% date) and a table of contents.
	%  All sections and subsections of the provided grading rubric must be present and should include
	% an introduction.
	% Validation (20 points)
	% - Chronogramme de la cellule-type 5 points
	% - Résultat du Testbench 10 points
	% - Implémentation matérielle 
	%   Théorie et Design (40 points)
	% - Définition d'une cellule-type 3 points
	% - Définition d'un signal inter cellule 2 points
	% - Schéma de votre cellule-type et description des signaux 5 points
	% - Définition d'une table de vérité 2 points
	% - Table de vérité de votre cellule-type 5 points
	% - Définition d'une table de Karnaugh et de son fonctionnement 3 points
	% - Table de Karnaugh de votre cellule-type 5 points
	% - Comment créer un composant par composition 3 points
	% de plusieurs cellule-types
	% - Explication des "boundary signal" 2 points
	% - Schéma complet de votre comparateur 7 points
	% - Déduction des liens de connexions de la ieme cellule-type 
	% Résultat et Discussion (25 points)
	% - Chronogramme de votre cellule type et discussion 5 points
	% - Discussion sur les résultats du fichier data.out 3 points
	% - Discussion sur les chronogrammes de test bench montrant :
	% o Le fonctionnement d'une comparaison +/+ 3 points
	% o Le fonctionnement d'une comparaison -/- 3 points
	% o Le fonctionnement d'une comparaison +/- 3 points
	% o Le fonctionnement d'une d'une comparaison -/+ 3 points
	% - Résultat implémentation matérielle 5 points
	% o For the section on the timing diagram of your typical cell and discussion: Please identify the 4
	% sections (sign bit, PP, PG, EG) and analyze each of these sections of the timing diagram
	% explaining whether you obtain the correct results and why according to your implemented logic.
	% o For the section concerning the i-th typical cell: Explain how you obtained the
	% equations of your iterative circuit by representing a diagram for a typical cell
	% with index i and the indices of the inputs and outputs of this typical cell as a function of i.
	% o For the section concerning the data.out file: Discuss your results but also
	% indicate how the file was generated and the strategy used.
	% o For the section discussing the test bench timing diagrams: You must present your
	% Modelsim timing diagrams and explain the validity of the results for the different
	% cases of comparisons of positive and negative numbers. o For the hardware results section: Explain how you performed the
	% tests with the FPGA, all the steps (e.g., pin assignment, block diagram, choice of LEDs/switches and what they represent) and also include a
	% summary table that demonstrates the operation of the FPGA for the 4 cases (sign bit,
	% PP, PG, EG), knowing that the switches are your inputs and the LEDs your
	% outputs. • Each figure and table must include a numbered caption that describes it (at the top for tables
	% and at the bottom for figures). Remember to indicate at the top of your truth tables which
	% variables are the inputs and outputs.
	\section{Résultat et discussion(Louis-Jean)}
	\paragraph{}
	Cette section sera sujet de nos résultats ainsi que de nos conclusions faisant référence à ceux-ci. Pour visualiser nos résultats, nous devions utiliser des	chronogrammes afin de pouvoir illustrer l’évolution temporelle de nos entrées et sorties.\\
	\section{Chronogramme de cellule-type (Louis-Jean)}
	
	
	\subparagraph{}

Dans celui de la cellule type, généré avec Waveform, il faut qu’il soit représentatif de la table de vérité de la cellule-type. Étant donné le chronogramme mis ci-dessous, nous avons donc eu une bonne représentation graphique de notre table de vérité montrant les sorties Xout et Yout étant à 1.
	
	\begin{figure}[H]
		\centering
		\includegraphics[width=0.7\linewidth]{"2025-10-29 14 22 27"}
		\caption{}
		\label{fig:2025-10-29-14-22-27} Chronogramme de la cellule-type
	\end{figure}
	\newpage
\section{Chronogramme de Test Bench (Louis-Jean)}
\paragraph{}
Dans le chronogramme du test Bench, il faut qu’il soit représentatif du circuit dans tout son entièreté. Il faut donc se fier aux équations de sorties trouver dans les tables de Karnaugh des signaux d’intercellules.\\\\
Nous voyons  aussi  dans le chronogramme les fonctionnements de comparaison :\\\\
\begin{itemize}
	\item Le fonctionnement d’une comparaison  +/+ : 
	Le fonctionnement d'une comparaison +/+ : Veux dire que l'on compare directement la valeur de A et B\\\\
	\item Le fonctionnement d’une comparaison -/- :  Veux dire que l’on inverse la valeur de A et B et ensuite on les compare~\\\\
	\item Le fonctionnement d'une comparaison +/- : Veux dire que la valeur de A ou B est inversé et puis on la
	compare à la valeur de l'autre entrée\\\\
\end{itemize}
Ainsi, étant donné le chronogramme mis ci-dessous, nous avons donc eu une bonne représentation graphique de notre table de vérité montrant les sorties PGout, PPout et EGout. \\\\

\begin{figure}[H]
	\centering
	\includegraphics[width=0.7\linewidth]{"2025-10-29 15 32 39"}
	\caption{}
	\label{fig:2025-10-29-15-32-39}Chronogramme du textbench
\end{figure}




\section{Fichier data.out (Louis-Jean)}
\subparagraph{} Le fichier data.out a été généré par un banc  d’essai VHDL. Cependant, pour générer ce fichier, il faut tout d’abord compiler les fichiers VHDL de notre design en utilisant des scripts que nous avons complétés. Ensuite il faut lancer une simulation dans la fenêtre de commande dans le logiciel Modelsim. En comparant les entrées A et B avec les signaux intercellules, nous avons un résultat à la fin de chaque ligne qui nous indique si l’on a réussi. Dans notre cas, nous avons juste eu des réussites. Ceci veut dire que la procédure a fonctionné.


\section{Résultat implémentation matérielle (Louis-Jean)}
\subparagraph{}Pour réaliser au niveau matérielle notre design, nous avons dû l’implémenter dans un circuit programmable  FPGA. Pour l’implémenter il fallait : 

\begin{enumerate}
	\item Reproduire le schéma bloc donné dans le laboratoire
	\item Assigner les broches d’entrées, d’interrupteurs et de DEL dans un fichier QSF
	\item Configurer le FPGA  dans Quartus selon la procédure
	\item Télécharger le design dans le FPGA
\end{enumerate}
\section{Conclusion (Louis-Jean)}
\paragraph{}
Ce laboratoire a permis de concevoir  un comparateur  4 bits en complément à deux.  Nous avons d’abord effectué une préparation avant chaque séance de laboratoire afin que nous puissions comprendre le fonctionnement général des circuits itératif. Après cette préparation, nous avons pu écrire en VHDL nos circuits planifiés. La simulation de notre design nous a permis de visualiser les entrées et les sorties que nous avions déclarées afin valider la fonctionnalité de notre design. Enfin, l’implémentation de notre circuit en FPGA nous a permis de valider en conditions réelles la logique conçue en VHDL. Malgré le bon déroulement du laboratoire, nous éprouvions des difficultés à ouvrir les fichiers du laboratoire sur des postes autre que l’ordinateur source. Ceci est probablement dû à l’utilisation des outils infonuagiques.

\end{document}



